\documentclass{article}

\usepackage{graphicx}
\usepackage{epstopdf}

\title{A Program to Estimate Resolution for Charged Particles in GlueX \\
~\\
\large GlueX-doc-1015-v2
}
\author{Mark M.\ Ito, Dmitry A. \ Romanov \\
Thomas Jefferson National Accelerator Facility \\
12000 Jefferson Avenue \\
Newport News, VA 23606
}
\date{June 15, 2012}

% global parameters
\textwidth=6.5in
\oddsidemargin=0in % use built-in offset of 1 inch for left margin
\evensidemargin=0in % ditto for even pages
\textheight=9in
\topmargin=0in
\headheight=0in % no headers in this document
\headsep=0in

\begin{document}

\bibliographystyle{plain}

\maketitle

\begin{abstract}
This note contains a documentation of the CCDB package. A package for storing and managing calibration constants database.
\end{abstract}

\tableofcontents

\section{Introduction}


\begin{figure}
  % Requires \usepackage{graphicx}
  \includegraphics[width=\textwidth]{ccdb_overall_tasks.eps}\\
  \caption{CCDB area of usage}\label{fig:ccdb_overall_tasks}
\end{figure}

CCDB stores data as tables with columns and rows. CCDB supports:
\begin{itemize}
\item \textbf{Naming}. Each table is identified by path-name;
\item \textbf{Versioning}. Each table may has many versions of data;
\item \textbf{Branching}. So called "variations" allows to use branches of data;
\end{itemize}

As a management tool and as a data provider CCDB allows:
\begin{itemize}
\item C++ User API. Allows an easy access to CCDB data from C++.
\item JANA API. An integration to JANA framework.
\item Python API. Allows acessing and managing CCDB from python.
\item Command line tools. Tools to manage CCDB data from the shell.
\item Web interface.
\end{itemize}
An illustration of this list one can see on pic \ref{fig:ccdb_overall_tasks}.

\section{Basic concepts}\label{sec:using}


\small
\begin{verbatim}
      SUBROUTINE REZEST_FDC_CDC(P, LAMBDA, M,
     X     DP_OVER_P, DPHI_TOT, DTHETA_TOT)
CCCCCCCCCCCCCCCCCCCCCCCCCCCCCCCCCCCCCCCCCCCCCCCCCCCCCCCCCCCCCCCCCCCCCCCC
C
C This routine estimates the resolution in GlueX for charged particles
C in tranverse momentum, azimuthal angle, and polar angle.
C
C Input arguments, all REAL*4
C
C     P        Magnitude of total momentum (GeV/c)
C     LAMBDA   Dip angle, difference in polar angle in lab between track
C              and pi/2 (i. e., 90 degrees) (radians)
C     M        Mass of the particle (GeV/c^2)
C
C Output arguments, all REAL*4
C
C     DP_OVER_P  Relative resolution in transverse momentum
C                ("sigma_{p_t}/p_t")
C     DPHI_TOT   Resolution in azimuthal angle ("sigma_phi")
C     DTHETA_TOT Resolution in polar angle ("sigma_theta")
C
C The routine combines the measurements in the FDC and CDC where
C appropriate. Parameters describing the geometry and materials are
C defined in the routine REZEST_COMPONENTS which appears below.
C
\end{verbatim}
\normalsize

\subsection{Getting the code}

Two methods:

\begin{enumerate}
\item Get the tar ball from
\begin{center}
{\tt http://www.jlab.org/$\sim$marki/misc/rezest.tar}
\end{center}
\item Check it out from the subversion repository with the command
\begin{center}
{\tt svn checkout
 https://halldsvn.jlab.org/repos/trunk/home/marki/gluex/rezest}
\end{center}
\end{enumerate}

\subsection{Building the files}

There is a simple makefile in the directory:

\begin{verbatim}
> cd <rezest directory>
> make
gfortran -g   -c -o rezest.o rezest.F
gfortran -g   -c -o rezest_fdc_cdc.o rezest_fdc_cdc.F
ar rcv librezest.a rezest.o rezest_fdc_cdc.o
a - rezest.o
a - rezest_fdc_cdc.o
gfortran -g   -c -o rezest_point.o rezest_point.F
gfortran -o rezest_point rezest_point.o -L. -lrezest
gfortran -g   -c -o rezest_point_comp.o rezest_point_comp.F
gfortran -o rezest_point_comp rezest_point_comp.o -L. -lrezest
\end{verbatim}

This creates three files that you care about:

\begin{enumerate}
\item {\tt librezest.a}: the object library
\item {\tt rezest\_point}: a binary
\item {\tt rezest\_point\_comp}: a binary
\end{enumerate}

\subsection{Using the files}

\section{Conclusions}

The plots show reasonable agreement with the HDGEANT
results. Agreement is generally at the 20\% level, in some places
better, in others as poor as a factor of 2.

One area where the simple model can breaks down is in the
straight-line approximation for the trajectories for particles with
very low transverse momentum. As a result predictions for extreme
forward angles are suspect. We have already pointed out a problem with
this approximation in estimating the contribution of curvature
resolution to azimuthal angular resolution in
Section~\ref{sec:curve-angle}.  Also since the measurement are assumed
to be equally spaced in both the FDC and the CDC, some of the features
in resolution visible in the transition polar angle region between the
two detectors is not reproduced; the real detector does not have a
smooth loss of CDC hits and a smooth gain of FDC hits as the polar
angle moves forward as does the model used in the estimates.

The most profitable use of these routines is probably not in
predicting the absolute level of resolutions in the detector, but in
predicting relative changes in resolution as detector parameters are
changed. The former requires a more detailed modeling of the detector
but also requires a greater effort whenever a new design is
proposed. This resolution estimator ({\tt REZEST}) is useful in
exploring the parameter space during the optimation process.

\appendix

\section{Some useful equations}\label{app:algebra}


\bibliography{ccdb_doc}

\end{document}
